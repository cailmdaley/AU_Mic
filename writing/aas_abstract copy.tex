
\documentclass[12 pt, letterpaper]{article}
% \usepackage[margin=1in]{geometry}

\begin{document}
\abstract
Disks of optically thin debris dust surround $\geq$ 20\% of main sequence stars and mark the final stage of planetary system evolution. The features of debris disks encode dynamical interactions between the dust and any unseen planets embedded in the disk.  The vertical distribution of the dust is particularly sensitive to the total mass of planetesimal bodies in the disk, and is therefore well suited for constraining the prevalence of otherwise unobservable Uranus and Neptune analogs. Inferences of mass from debris disk vertical structure have previously been applied to infrared and optical observations of several systems, but the smaller particles traced by short-wavelength observations are ‘puffed up’ by radiation pressure, yielding only upper limits on the total embedded mass. The large grains that dominate the emission at millimeter wavelengths are essentially impervious to the effects of stellar radiation, and therefore trace the underlying mass distribution more directly. Here we present 1.3mm dust continuum observations of the debris disk around the nearby M star AU Mic with the Atacama Large Millimeter/submillimeter Array (ALMA).  The 3 au spatial resolution of the observations, combined with the favorable edge-on geometry of the system, allows us to measure the vertical structure of a debris disk at millimeter wavelengths for the first time. We analyze the data using a ray-tracing code that translates a 2-D density and temperature structure into a model sky image of the disk. This model image is then compared directly to the interferometric data in the visibility domain, and the model parameters are explored using a Markov Chain Monte Carlo routine. We measure a scale height-to-radius ratio of 0.03, which we then compare to a theoretical model of steady-state, size-dependent velocity distributions in the collisional cascade to infer a total mass within the disk of ∼ 1.7 Earth masses.  These measurements rule out the presence of a gas giant or Neptune analog in the outer disk, but are suggestive of the presence of large planetesimals required to stir the dust distribution.
\end{document}
