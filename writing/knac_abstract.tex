
\documentclass[12 pt, letterpaper]{article}
% \usepackage[margin=1in]{geometry}

\begin{document}
\abstract
  We present the preliminary results of our analysis of Atacama Large Millimeter/submillimeter Array (ALMA) 1.3~mm dust continuum observations of the debris disk around nearby M star AU Mic.
  Debris disks, dusty and optically thin clouds accompanying $\geq 20\%$ of stars, figure as the final stage of planetary system evolution. 
  Planets are known to interact with their parent disks throughout the debris disk phase, producing observable disk features that can help clarify the entangled evolutionary paths of planets and circumstellar disks. 
  These observables may also be used to provide evidence for the existence of Neptune-sized planets not directly observable by other means. 
  AU Mic's proximity ($9.91 \pm 0.10$~pc) and edge-on inclination combine with the excellent 0.3~au resolution of the ALMA observations to provide a favorable opportunity to measure the disks's vertical scale height. 
  This quantity is of interest because the total mass of all bodies in the disk can be derived by relating the scale height to the size-dependent particle velocity distribution induced by the largest bodies dynamically stirring the disk. 
  Using Markov Chain Monte Carlo routines, we derive a total mass of 1.68 Earth masses; considering that we derive a dust mass of only 0.01 Earth masses, our analysis suggests the presence of large planetesimals or even a planet.
  While estimates of the total disk mass have previously been derived in this manner using infrared and optical observations, the smaller particles traced by such observations are `puffed up' by radiation pressure, leading to overestimates of the mass. 
  As such, our estimate based on mm-wavelength observations is the first of its kind.
\end{document}
