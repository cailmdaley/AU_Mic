
\documentclass[12 pt, letterpaper]{article}
% \usepackage[margin=1in]{geometry}

\begin{document}
\abstract
  Debris disks, dusty and optically thin clouds surrounding $\geq 20\%$ of main sequence stars, figure as the final stage of planetary system evolution. 
  Planets interact with the dust in which they are enveloped throughout the debris disk phase, producing observable disk features that can help clarify the entangled evolutionary paths of planets and circumstellar disks. 
  Observables such as the vertical distribution of the dust may also be used to constrain the total mass of planetesimal bodies in the disk and thus provide evidence for the existence of Neptune-sized planets not directly observable by other means. 
  This technique has been applied to infrared and optical observations, but the smaller particles they trace are `puffed up' by radiation pressure leading to overestimates of the total mass. 
  As such, measurements must be performed at longer wavelengths. 
  In this light, we present an analysis of Atacama Large Millimeter/submillimeter Array (ALMA) 1.3~mm dust continuum observations of the debris disk around nearby M star AU Mic at a spatial resolution of 0.3 au.
  AU Mic's proximity ($9.91 \pm 0.10$~pc) and edge-on inclination provide a favorable opportunity to measure the disks's vertical scale height, which can be related to the size-dependent particle velocity distribution induced by the largest bodies dynamically stirring the disk in order to estimate the total mass of bodies in the disk. 
  To constrain the scale height, we use a ray-tracing code that translates a 2-D density and temperature structure into a model sky image of the disk. 
  This model image can then be Fourier transformed and compared directly to the observed interferometric data, allowing analysis of the model properties using a Markov Chain Monte Carlo routine.
  We derive a total mass of $\sim1.7$ Earth masses; considering that we derive a dust mass of only $\sim0.01$ Earth masses, these measurements rule out the presence of a Neptune-sized planet but are suggestive of large planetesimals.
\end{document}
