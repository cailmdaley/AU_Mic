\documentclass[12 pt, letterpaper]{article}
\usepackage[
  separate-uncertainty=true,
  multi-part-units=single, 
  bracket-numbers=false]{siunitx}
\usepackage{amsmath, amssymb}
\newcommand{\Mearth}{M_\oplus}

\begin{document}
\abstract
Disks of optically thin debris dust surround $\geq$ 20\% of main sequence stars and mark the final stage of planetary system evolution. 
The features of debris disks encode dynamical interactions between the dust and any unseen planets embedded in the disk.  
The vertical distribution of dust in debris disks is particularly sensitive to the total mass of planetesimal bodies stirring the disk, and is therefore well suited for constraining the prevalence of otherwise unobservable Uranus and Neptune analogs. 
Inferences of dynamical mass from debris disk vertical structure have previously been applied to infrared and optical observations of several systems, but the smaller particles traced by short-wavelength observations are `puffed up' by radiation pressure, yielding only upper limits on the total dynamical mass. 
The large grains that dominate the emission at millimeter wavelengths are essentially impervious to the effects of stellar radiation, and therefore trace the underlying mass distribution more directly. 
Here we present ALMA \SI{1.3}{mm} dust continuum observations of the debris disk around the nearby M star AU Mic.  
The \SI{3}{au} spatial resolution of the observations, combined with the favorable edge-on geometry of the system, allows us to measure the vertical structure of the disk; we report a scale height-to-radius aspect ratio of $h = \num{0.031 \pm 0.005}$. 
Comparing the observed aspect ratio to a theoretical model of steady-state, size-dependent velocity distributions in the collisional cascade we estimate the total mass of bodies stirring the disk to be $\sim \SI{1.5}{\Mearth}$.  
These measurements rule out the presence of a gas giant or Neptune analog in the outer disk, but are suggestive of the presence of large planetesimals stirring the dust distribution.
\end{document}
