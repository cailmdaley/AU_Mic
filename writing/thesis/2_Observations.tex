
\documentclass[12pt,modern,oneside]{book}

 % Double spacing
\usepackage{setspace}
\setstretch{2}

% Nice tables
\usepackage{booktabs}

\begin{document}


\title{AU Mic}

\author{Cail Daley}


\section{Observations}

AU Mic was observed on three dates with ALMA: 26 March 2014, 18 August 2014, and
24 June 2015. All observations were configured with four spectral windows, and
employed ALMA's 12 m antennas and Band 7 receivers. One spectral window was
centered around the CO $J = (2-1)$ transition at a frequency of 230.538001 Ghz,
while the remainging three were configured to detect continuum emission with
maximum bandwidths of 2 Ghz and channel spacing of 15.6 Mhz. Central frequencies
for the continuum bands are 228.5, 213.5, and 216.0 Ghz.

The 26 March data were obtained with 32 antennas and baselines ranging between
14 and 437 m; weather conditions were excellent ($\sim$0.6 mm of precipitable
water vapor (PWV)). The quasar J1924-2914 was used as a bandpass calibrator, and Titan
was used to calibrate absolute flux. After these initial calibrations,
observations cycled every seven minutes between AU Mic and the quasar
J2101-2933, which was used for phase calibration. In total, 35 minutes were
spent on source.

The 18 August utilized 35 antennas in a more extended antenna configuration
(baselines between 20 and 1268 m) to probe the small scale structure of the
disk. Weather conditions were poor, with $\sim$1.6 mm PWV. The quasars J2056-4714 (bandpass calibration) and J2056-472 (absolute
flux calibration) were observed at the beginning of the observation window. For
the remainder of the time block antennas cycled between seven-minute
observations of AU Mic and brief observations of the quasars J2101-2933, for
phase calibration, and J2057-3734, to test the quality of the gain transfer. AU
Mic was observed for 35 minutes altogether.

The 24 June observation was taken to supplement the August observation, which was of
poor quality due to weather conditions. 37 antennas covered baselines between 30 and
1431 m, and weather conditions were good: $\sim$0.7 mm PWV. Bandpass and absolute flux calibrations, making use of J1924-2914 and
Titan respectively, were conducted at the beginning of the scheduled time block.
Short observations of the quasars J2056-3208 for phase calibration and
J2101-2933 to assess gain transfer quality were interspersed among
seven-minute observations of the source, which was observed for 33 minutes. The
host star flared during the last observation of AU Mic, from 04:23:38-04:29:58
UT.

Calibration, reduction, and imaging were carried out using the \texttt{CASA} and
\texttt{MIRIAD} software packages. Standard ALMA reduction scripts were applied
to the datasets: phase calibration was accomplished via water vapor radiometry
tables, and system temperature calibrations were performed to account for
variations in instrument and weather conditions. Flux and bandpass calibrations
were subsequently applied.

The authors travelled to the NRAO facility in Charlottesville, VA in October
2015 to further process the data; in particular the trip was intended to allow
on-site correction of the 24 June flare. Tasks used to reduce the data at the
NRAO facility were all part of the CASA package. An elliptical gaussian was fit
to a small region around the star in the image plane of each dataset using the
task \texttt{imfit}; the equatorial coordinates of the the model gaussian
centroid were used to define the star position. Each observation was then phase
shifted using the task \texttt{fixvis} so that the the pointing center was the
same as the star position fit. The peak flux of the model gaussian
was also subtracted from the location of the star in the visibility domain so
that only the disk remained.

The 24 June dataset required additional reduction due to the flare. While for
the other dates we were able to fit a single point source to account for the stellar
component over the entire observation, the flare required that the dataset be
split into one minute bins between 04:23:38 and 04:29:58 in order to account for
the variable flux of the host star. For each of these bins, we used the task
\texttt{uvmodelfit} to fit a point source to the long baseline visibilities,
which we subsequently subtracted from each bin in the visibility domain.
Subtracted point fluxes can be found in Table \ref{tab:flare fluxes}.

\begin{table}
	\caption{Subtracted point source fluxes \label{tab:flare fluxes}}
	\begin{center}
		\begin{tabular}{cr}
		  \tablehead{
		  \colhead{Time (UTC)} &
		  \colhead{Point source flux ($\mu$Jy)}
		  }
		  \startdata
		  03:45:0-04:20:00 (no flare) & $410 \pm\ \ 20$\\
			4:23:38-4:24:00 & $920 \pm 170$ \\
			4:23:38-4:24:00 & $11,460 \pm 100$ \\
			4:25:00-4:26:00 & $3590 \pm 100$ \\
			4:26:00-4:27:00 & $1580 \pm 100$ \\
			4:27:00-4:28:00 & $450 \pm 100$ \\
			4:28:00-4:29:00 & $460 \pm 100$ \\
			4:29:00-4:29:58 & $520 \pm 100$ \\
		  \enddata
		\end{deluxetable}
	\end{center}
\end{table}

The June date pointing center, defined by the centroid of the elliptical
gaussian fit to the star, was visibly offset from the surrounding disk; this
could be explained if the flare referenced above were not symmetric with respect
to the star. The offset was remedied by redefining the pointing center as
follows. Because the star is known to be located at the center of the disk, we
can use information provided by the brightness distribution of the disk to infer
the star position. We do so by selecting the brightest pixel on each side of the
disk from the clean component map (the \texttt{.model} file produced by
\texttt{tclean}), and redefining the star/pointing center as the mean of the two
pixel positions. This yields offsets of $(0.01'', -0.05'')$ for the March
observation, $(0.01'', 0.00'')$ for the August observation, and $(0.00'',
0.09'')$ for the June observation. Given the good agreement between the
calculated star position and the image center for the two non-flare dates (March
and August), we conclude that the `pixel' method represents a viable way to
accurately determine star position. We apply this correction and redefine the
image center via \texttt{fixvis}. For consistency, we apply the phase shift to
all three dates.

Due to the high proper motion of AU Mic, the pointing centers of the three dates
differ by a not-insignificant amount. When datasets with different pointing
centers are cleaned together with \texttt{tclean}, the pointing center of the
first dataset is taken as the new pointing center, and the data are combined in
in the $uv$ plane with each subsequent dataset offset from the first as given by
their relative pointing centers. In the case of AU Mic, this leads to an  image
of the disk composed of three observations offset with respect to each other. To
remedy this, we use the task \texttt{concat}, which combines datasets with their
pointing centers aligned so long as the pointing centers do not differ by more
than the value of \texttt{dirtol}. We set \texttt{dirtol} to $ 2''$, a value
larger than  AU Mic's proper motion over ALMA's $\sim 1$ year observation
baselines.


% \begin{deluxetable}{lr}	
% 	\tablecaption{Subtracted point-source fluxes \label{tab:flare fluxes}}
%   \tablehead{
%   \colhead{Time (UTC)} &
%   \colhead{Point-source Flux ($\mu$Jy)}
%   }
%   \startdata
%   03:45:0-04:20:0 (no flare) & ($4.1 \pm 0.2)  \times 10^2$\\
% 	4:23:38-4:24:00 & $(9.2 \pm 1.7) \times 10^2$ \\
% 	4:23:38-4:24:00 & $(1.146 \pm 0.010) \times 10^4$ \\
% 	4:25:00-4:26:00 & $(3.59 \pm 0.10) \times 10^3$ \\
% 	4:26:00-4:27:00 & $(1.58 \pm 0.10) \times 10^3$ \\
% 	4:27:00-4:28:00 & $(4.50 \pm 1.0) \times 10^2$ \\
% 	4:28:00-4:29:00 & $(4.60 \pm 1.0) \times 10^2$ \\
% 	4:29:00-4:29:58 & $(5.20 \pm 1.0) \times 10^2$\\
%   \enddata
% \end{deluxetable}
% \clearpage

% \clearpage
% \bibliographystyle{aasjournal}
% \bibliography{AGN_Feedback_Paper}


\end{document}
