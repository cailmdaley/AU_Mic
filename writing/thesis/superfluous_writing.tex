OBSERVATIONS
% The peak flux of the model gaussian was also subtracted from the location of the star in the visibility domain so that only the disk remained.

% The 24 June dataset required additional reduction due to the flare. While for
% the other dates we were able to fit a single point source to account for the stellar
% component over the entire observation, the flare required that the dataset be
% split into one minute bins between 04:23:38 and 04:29:58 in order to account for
% the variable flux of the host star. For each of these bins, we used the task
% \texttt{uvmodelfit} to fit a point source to the long baseline visibilities,
% which we subsequently subtracted from each bin in the visibility domain.
% Subtracted point fluxes can be found in Table \ref{tab:flare fluxes}.

% The June date pointing center, defined by the centroid of the elliptical
% gaussian fit to the star, remained visibly offset from the disk; this
% could be explained if the flare referenced above were not symmetric with respect
% to the star. The offset was remedied by redefining the pointing center as
% follows. Because the star is known to be located at the center of the disk, we
% can use information provided by the brightness distribution of the disk to infer
% the star position. We do so by selecting the brightest pixel on each side of the
% disk from the clean component map (the \texttt{.model} file produced by
% \texttt{tclean}), and redefining the star/pointing center as the mean of the two
% pixel positions. This yields offsets of $(0.01'', -0.05'')$ for the March
% observation, $(0.01'', 0.00'')$ for the August observation, and $(0.00'',
% 0.09'')$ for the June observation. Given the good agreement between the
% calculated star position and the image center for the two non-flare dates (March
% and August), we conclude that the `pixel' method represents a viable way to
% accurately determine star position. We apply this correction and redefine the
% image center via \texttt{fixvis}. For consistency, we apply the phase shift to
% all three dates.

% Due to the high proper motion of AU Mic, the pointing centers of the three dates
% differ by a not-insignificant amount. When datasets with different pointing
% centers are cleaned together with \texttt{tclean}, the pointing center of the
% first dataset is taken as the new pointing center, and the data are combined in
% in the $uv$ plane with each subsequent dataset offset from the first as given by
% their relative pointing centers. In the case of AU Mic, this leads to an  image
% of the disk composed of three observations offset with respect to each other. To
% remedy this, we use the task \texttt{concat}, which combines datasets with their
% pointing centers aligned so long as the pointing centers do not differ by more
% than the value of \texttt{dirtol}. We set \texttt{dirtol} to $ 2''$, a value
% larger than  AU Mic's proper motion over ALMA's $\sim 1$ year observation
% baselines.
