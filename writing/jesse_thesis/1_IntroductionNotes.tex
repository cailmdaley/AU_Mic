\chapter{Introduction}
%%Characterizing the transitional period from gas-rich protoplanetary disks to tenuous, nearly gas-free debris disks is imperative for understanding how planetary systems form. This transitional period is  distinguished by an inside-out clearing of sub-mm dust from the star and dispersal of gas from the system. The gas in protoplanetary disks generally disperses within 10 million years of stellar formation (Mamajek 2009), meaning that there are very few targets that occupy this transitional stage. 49 Ceti breaks both of these conventions, however. It is one of only a handful of systems ($\beta$ Pic, HD 21997, and HD 141569; Dent et al. 2014, Moor, et al. 2013, Weinberger et al. 2000) that display the dust properties of optically thin debris disks but retain enough molecular gas to be detectable at millimeter wavelengths. The dust distribution of 49 Ceti was first investigated with mid-infrared imaging (Wahhaj et al. 2007). These results suggested small grains (a\textasciitilde1$\mu$m) between 30AU and 60AU from the star are responsible for the majority of mid-infrared excess, whereas an extended disk of larger grains (a\textasciitilde15$\mu$m) is responsible for the rest of the infrared excess. Hughes et al. (2008) resolved the gas disk around 49 Ceti for the first time with the Sub-Millimeter Array (SMA), revealing CO in Keplerian rotation between 40 and 200AU, including a central clearing, potentially consistent with photoevaporation. At the time, 49 Ceti was thought to be 7.8 Myr old based on its position on the HR diagram (Thi et al, 2001), but its age is uncertain due to its isolation. However, this estimate was revised to \textasciitilde40 Myr with the classification of 49 Ceti as part of a moving group (Zuckerman \& Song 2012), which makes it the oldest known main sequence star to contain a substantial reservoir of molecular gas. Here we present ALMA observations of 49 Ceti that resolve the extended dust disk for the first time, helping to complete the picture of the system and probe the processes by which the gas and dust disperse.

...49 Cet, beta pic, HR 4796A have large infrared excesses
HIP7345
HD9672

Sadakane & Nishida 1986 (bibd)
first identified as a "candidate vega-like" source by Sadakane & Nishida (1985) when they searched the IRAS point-source catalog for Bright Star Catalog (BSC) members with significant 60 micron excesses. 

Bockelee-Morvan et all 1994 (in bib.bib)
IRAM 1.2mm flux of 12.7$\pm$2.3mJy

Jura et al 1993 (bibd)
Correlated IRAS point source found 49 Cet, beta pic, and HR 4796 are only 3 A stars in the BSC with tau > 10^-3. (went one step further than Sadakane)... these three have values of F(60 microns)/F(V band) > 0.1, which is what implies tau > 10^-3. 

Song et al 2004 << how to cite??? 
JCMT/SCUBA 850 micron flux of 8.2$\pm$1.9mJy. 
Inconsistent 

ZUCKERMAN 1995 (in bibbib) 
Marginal detection of CO J=2-1 emission at 1mm with IRAM (Institut de Radio-Astronomie Millimetrique) on Pico Veleta, Spain. Estimated 0.02-1 earth masses of millimeter-sized dust and an age of 10 million years old. Gave highly uncertain 140AU from star as outer radius of molecular gas disk. FIRST DETECTION OF CO (IN J2-1) THIS IS A SINGLE DISH OBS

COULSON 1998 (bibd)
Very low limit on the gas mass in the disk (less than 6x10^-8 solar masses) and estimate 5.0x10^-7 solar masses of dust. Used 3.8m UKIRT for mid and near IR spectroscopy and sub-mm observations made with JCMT. Derived gas masses Set gas to dust ratio at less than 0.12. 

WEINBERGER 1999(bib'd)
HST/NICMOS chronograph observations didn't see any scattered light at r>1.6 arcsec

THI 2001(a) (can/'t get original!!)
Marginal detection of molecular gas (\textasciitilde10^${-3}$ solar masses) using observations of the pure rotational transitions of H2. considered gas poor based on CO observations using H2/CO conversion factor of 10^4. 

THI 2001(b) (bib'd)
determined 49 Ceti's age to be 7.8Myr based on evolution tracks derived from Siess et al 2000, which take the accretion history into account. revises estimate of gas mass to be 3.5pm1.9 x 10^-3 solar masses. however, disks are not resolved, so the location of the gas is unknown. 

JAYA2001 (inbib)
Used Keck II to look at IR excess of 49 ceti. 49 Yeti is one of only three known with L(ir)/L(bol) > 10^-3. (the other two being beta pic and hr 4796). They marginally resolved the dust disk and suggest upper limits of 92AU (if p=0) and 1065AU (if p=1) for the outer radius of the dust disk. p here isn't power law, it is a measure of the grain emissivity... p=1 corresponds to small grain emission, and p=0 corresponds to blackbody emission from large grains. With constraints only set by their imaging, they rule out small grains, and derive a radius of ~90AU for the dust disk. However, they are greatly simplifying (no distr of grain sizes, no diff compositions). They also say "the observed source size is consistent with a r~50 AU (with p~0) disk. (Dent 2005 words this as "Jaya did detect midIR emission extending to a radius of ~50AU). 

DENT2005 (bibd) THIS IS SINGLE DISH OBS
Surveyed visible stars with IR excesses with JCMT at J=3-2 12CO emission. Their data are consistent with a compact disk with outer radius ~17AU and inclination 16deg or a more extended ring of radius 50AU and inclined ring of i~35deg. From their data, they suggest the latter is consistent with the location of the dust, but fails to recreate the high-velocity double-peaked structure apparent in the CO line profile. 

CHEN 2006 (bibd)
Spitzer space telescope IRS 5.5-35 micron spectra
no emission features associated with micron sized grains in the mid-IR is seen
used in SED modeling

WAHHAJ 2007 (bibd)
...The dust distribution of 49 Ceti was first investigated with mid-infrared imaging \citep{Wahhaj07}(Wahhaj et al. 2007).
...Detected dust at 12.5 and 17.9 microns extended along a NW-SE axis with an inclination of 60 deg
...Resolution of less than an arc second with Keck II observed in 1999 
...Modeled a flat optically thin disk, limited number of param because limited spatial resolution and not a lot of flux density measurements (at time of paper)
...Constrained the temp and size of grains by the relative brightness of the disk between the two measurements
...A single disk model was unable to produce the extent of the emission in the mid-IR images at 12.5 and 17.9 microns
...A two part disk was necessary to recreate data. Small grains (a\textasciitilde1$\mu$m) between 30AU and 60AU from the star recreate the mid-IR images, whereas an extended disk of larger grains (a\textasciitilde15$\mu$m) fit the SED. 
%These results suggested small grains (a\textasciitilde1$\mu$m) between 30AU and 60AU from the star are responsible for the majority of mid-infrared excess, whereas an extended disk of larger grains (a\textasciitilde15$\mu$m) is responsible for the rest of the infrared excess. 


RHEE 2007(biibd)
posit that CO emission suggests a young age, but galactic space motions UVW = (-23, -17, -4) wrt. the sun is not indicative of extreme youth... suggest an age of 20Myr. 

HUGHES 2008 (bibd)
is 49 yeti at the end of its primordial gas dissipation phase? 
inclination is 90pm5 deg.

MOOR 2011(bibd)
A3 type HD21997 debris disk shows sub-mm CO emission (cite this if you're gonna cite anything in intro)
neither primordial origin or steady secondary production from icy planetescimals can unequivocally explain the CO gas in HD21997
HD21997 and 49 Cet harbor the most extended disks, suggesting that large radius and low dust temp may be essential for CO detection
Both 21997 and 49's A type central stars UV radiation could dissociate CO molecules in the vicinity of the star... both harbor a large amnt of relatively cold dust (T<80K, Mass~0.1 earth masses)
primordial co gas can only survive if there is efficient shielding from the stellar/interstellar high-energy photons... for 49 Ceti, the lifetime is ~ 500 yr. 

ZUCKERMAN & SONG 2012(bibd0)
identified 49 CEti as part of the co-moving Argus association, revising its age to 40Myr based on galactic space motions (UVW) and its location on a color magnitude diagram 

ROBERGE 2013(bibd)
Detected dust emission at 70, 160, 250, 350, and 500 microns
Herschel far-IR imaging and spectroscopy at 70microns spatially resolved dust continuum emission from the outer disk for the first time-- didn't see a central clearing in the dust. Radius of dust disk is deemed ~200AU. 

KOSPAL 2013 (gas)(bibd) and MOOR 2013(continuum) (bibd)
HD21997 and 49 Ceti are the oldest members of the gaseous debris disk sample and contain the largest amount of cold gas-- helps us in consideration whether the gas is primordial or secondary


ROBERGE 2014(bibd)
HST STIS far-UV spectra found atomic absorption lines from gas along the line-of-sight to the central star
Determind LOS CI column density, estimate total carbon column density, and set limits on the OI column density
No LOS CO absorption-- assumed inclination was 90deg in this paper, we find it to be slightly inclined
C/Fe ratio is really high! greater than solar value, suggesting 49 Ceti is volatile-rich, similar to beta pics
Carbon overabundance may help trap dust that is smaller than the blowout size. SOURCE?

DENT 2014
along with 49 cet and HD21997, beta pic shows sub-mm CO emission in a debris disk(Cite this in intro if you mention it for beta pic)




.....

\section{These are interesting tidbits.}
There is no silicate emission in Spitzer Space Telescope observations, suggesting that the the circumstellar dust must be cooler than 200K (Kessler-silacci et al. 2005)... or that it's not silicates??

See WAHHAJ2007 page 7 for a discussion of how very small dust grains can survive around an A star. 

Most debris disks that are similar to 49 Ceti in terms of age and IR excess have never been observed in molecular lines (as most such candidates are in the southern hemisphere) (Moor 2011 in the HD21997 paper)


%\ifthenelse{\isodd{\thepage}}{\clearemptydoublepage}{}