\chapter{Summary and Conclusions}

We resolved 49 Ceti's outer disk in both continuum emission and CO for the first time using ALMA Cycle 1 observations at a wavelength of 850\,$\mu m$. To investigate the geometry of the dust disk and the size distribution of its constituent grains,  simultaneously modeled the ALMA visibilities and unresolved fluxes from the SED with three axisymmetric functional forms of the surface density: 1) a single power law, 2) a single power law with a narrow, spatially unresolved belt, and 3) a double power law. We found that all of these models were unable to recreate the data unless an inner belt with a smaller characteristic grain size was introduced. With this two-part disk model, we found that the inner disk, characterized by small, $\sim$ 0.1\,$\mu m$ grains, extends from $\sim$ 4 to $\sim$ 60\,AU, and that larger, $\sim$ 1.6\,$\mu m$ grains populate the outer disk, which extends from $\sim$ 60 to $\sim$ 300\,AU. 

The two-part disk models that account for a peak in the surface density distribution, i.e. the single power law with the spatially unresolved belt and the double power law, better reproduce the ALMA data than the single power law. These two models, which peak at 114$^{+3}_{-3}$\,AU and 100$^{+15}_{-14}$\,AU, respectively, are indistinguishable within our uncertainties. This suggests that the enhancement in the surface density is unresolved by our data, which have a resolution of $\sim$ 30\,AU. The peak in surface density suggests a region of enhanced collisions between planetesimals. \cite{Moor15} were the first to suggest that planetary stirring was responsible for the extent of the dust disk as derived from Herschel images ($\sim$ 250\,AU), but the density enhancement discovered in this work provides a better tracer for the radius at which active stirring takes place. Using the radius of this newfound ring, our analysis confirms this result. While we are unable to see the responsible planet directly, its signature is apparent in promoting planetesimal collisions and creating the millimeter-sized dust that we observe. 

%This ring is almost certainly stirred by the gravitational influence of an interior giant planet. T

There is a substantial mass of gas co-located with the millimeter-sized dust due to collisions between volatile-rich bodies and planetesimals. This secondary material is likely continually replenished in order to maintain the mass of dust and CO that we detect. The processes responsible for the distinct inner disk of small grains are not well understood, though breakup and sublimation of comets play a role \citep{Wahh07}. There is a disparity between the submicron-sized grains suggested for the inner disk by \citeauthor{Wahh07} and the lack spectral features in the IRS spectrum, one that can potentially be resolved if 49 Ceti's disk is dominated by carbonaceous grains rather than silicates. New scattered light observations with \textit{HST}/ACS will likely act as the tiebreaker here, as previous coronagraphic imaging may have failed to detect the disk simply due to sensitivity limitations and the distance of 49 Ceti rather than the low albedo of the grains. However, if future observations with \textit{HST}/ACS do not display any reflected dust, 49 Ceti's dust disk would be the lowest albedo system ever measured. 

% co-location of a substantial mass of CO with the millimeter-sized dust indicates collisions of volatile-rich bodies and planetesimals are responsible for maintaining the mass of the disk. If 

The unique morphology of 49 Ceti's dust disk and its substantial reservoir of molecular gas given its age are special features that help illuminate the late stages of disk evolution. Our ALMA data were able to resolve the outer dust disk at millimeter wavelengths for the first time, providing the means with which to trace the distribution and dynamics of the parent planetesimal belts. However, subtleties in the surface density distribution could have been washed out by our beam. For example, we can not tell the difference between the narrow ring and double power law descriptions of the surface density. In addition, we were unable to resolve the morphology of the inner disk, which could provide tantalizing clues to the location and mass of a possible planet. Future observations, with higher sensitivity and resolution, may be able to resolve these nuances. In addition, if the planet is at the separations we expect and maximum elongation, direct imaging is within the realm of possibility. While we have written a new chapter in understanding 49 Ceti and its implications for the evolution of circumstellar material, it is by no means a closed case. Advanced observations will almost certainly reveal new and exciting clues about the processes of planet formation.


\medskip
\bigskip\bigskip
\bigskip
This work was partially supported by NSF grant AST-1412647. This paper makes use of the following ALMA data: ADS/JAO.ALMA\#2012.1.00195.5. This research has made use of the SIMBAD database, operated at CDS, Strasbourg, France, and of NASA's Astrophysics Data System. 
