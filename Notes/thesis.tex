
\documentclass[11pt,modern]{aastex6}

\begin{document}


\title{AU Mic}

\author{Cail Daley}


\section{Observations}
AU Mic was observed on three dates with ALMA: 26 March 2014, 18 August 2014, and 24 June 2015. All observations were configured with four spectral windows, and employed ALMA's 12 m antennas and Band 7 receivers. One spectral window was centered around the CO $J = (2-1)$ transition at a frequency of 230.538001 Ghz, while the remainging three were configured to detect continuum emission with maximum bandwidths of 2 Ghz and channel spacing of 15.6 Mhz. Central frequencies for the continuum bands are 228.5, 213.5, and 216.0 Ghz.

The 26 March data were obtained with 32 antennas and baselines ranging between  14 and 437 m; weather conditions were excellent ($\sim$0.6 mm of precipitable water vapor). The quasar J1924-2914 was used as a bandpass calibrator, and Titan was used to calibrate absolute flux. After these initial calibrations, observations cycled every seven minutes between AU Mic and the quasar J2101-2933, which was used for phase calibration. In total, 35 minutes were spent on source.

The 18 August utilized 35 antennas in a more extended antenna configuration (baselines between 20 and 1268 m) to probe the small scale structure of the disk. Weather conditions were poor, with $\sim$1.6 mm of precipitable water vapor. The quasars J2056-4714 (bandpass calibration) and J2056-472 (absolute flux calibration) were observed at the beginning of the observation window. For the remainder of the time block antennas cycled between seven-minute observations of AU Mic and brief observations of the quasars J2101-2933, for phase calibration, and J2057-3734, to test the quality of the gain transfer. AU Mic was observed for 35 minutes altogether.

The 24 June observation was taken to supplement the 18 August's, which was of poor quality due to weather conditions. 37 antennas covered baselines from 30 to 1431 m and weather conditions were good: $\sim$0.7 mm of precipitable water vapor. Bandpass and absolute flux calibrations, making use of J1924-2914 and Titan respectively, were conducted at the beginning of the scheduled time block. Short observations of the quasars J2056-3208 for phase calibration and J2101-2933 to assess the quality of the gain transfer were interspersed among seven-minute observations of the source, which was observed for 33 minutes. The host star flared during the last observation of AU Mic, from 04:23:38-04:29:58 UT.

Calibration, reduction, and imaging were carried out using the CASA and MIRIAD software packages. Standard ALMA reduction scripts were applied to the datasets: phase calibration was accomplished via water vapor radiometry tables, and system temperature calibrations were performed to account for variations in instrument and weather conditions. Flux and bandpass calibrations were subsequently applied.

The authors travelled to the NRAO facility in Charlottesville, VA in October 2015 to further process the data; in particular the trip was intended to allow on-site correction of the 24 June flare. Tasks used to reduce the data at the NRAO facility were all part of the CASA package. An elliptical gaussian was fit to a small region around the star in the image plane of each dataset using the task \textit{imfit}; the equatorial coordinates of the the model gaussian peak were then used to phase shift the dataset via the task \textit{fixvis} to account for AU Mic's proper motion.\footnote{Should I put the derived phase centers in a table?} The peak flux of the model gaussian was also subtracted from the location of the star in the visibility domain so that only the disk remained.

The 24 June dataset required additional reduction due to the flare. While for other dates we were able to fit a single point source to account for the stellar component over the entire observation, the flare required that the dataset be split into one minute bins between 04:23:38 and 04:29:58 in order to account for the variable flux of the host star. For each of these bins, we used the task \textit{uvmodelfit} to fit a point source to the long baseline visibilities, which we subsequently subtracted from each bin in the visibility domain. Subtracted point fluxes can be found in Table \ref{tab:flare fluxes}.

We just split out the flare entirely in the end?

\begin{deluxetable}{cr}
	\tablecaption{Subtracted point source fluxes \label{tab:flare fluxes}}
  \tablehead{
  \colhead{Time (UTC)} &
  \colhead{Point source flux ($\mu$Jy)}
  }
  \startdata
  03:45:0-04:20:00 (no flare) & $410 \pm 20$\\
	4:23:38-4:24:00 & $920 \pm 170$ \\
	4:23:38-4:24:00 & $11,460 \pm 100$ \\
	4:25:00-4:26:00 & $3590 \pm 100$ \\
	4:26:00-4:27:00 & $1580 \pm 100$ \\
	4:27:00-4:28:00 & $450 \pm 100$ \\
	4:28:00-4:29:00 & $460 \pm 100$ \\
	4:29:00-4:29:58 & $520 \pm 100$ \\
  \enddata
\end{deluxetable}

% \begin{deluxetable}{lr}
% 	\tablecaption{Subtracted point-source fluxes \label{tab:flare fluxes}}
%   \tablehead{
%   \colhead{Time (UTC)} &
%   \colhead{Point-source Flux ($\mu$Jy)}
%   }
%   \startdata
%   03:45:0-04:20:0 (no flare) & ($4.1 \pm 0.2)  \times 10^2$\\
% 	4:23:38-4:24:00 & $(9.2 \pm 1.7) \times 10^2$ \\
% 	4:23:38-4:24:00 & $(1.146 \pm 0.010) \times 10^4$ \\
% 	4:25:00-4:26:00 & $(3.59 \pm 0.10) \times 10^3$ \\
% 	4:26:00-4:27:00 & $(1.58 \pm 0.10) \times 10^3$ \\
% 	4:27:00-4:28:00 & $(4.50 \pm 1.0) \times 10^2$ \\
% 	4:28:00-4:29:00 & $(4.60 \pm 1.0) \times 10^2$ \\
% 	4:29:00-4:29:58 & $(5.20 \pm 1.0) \times 10^2$\\
%   \enddata
% \end{deluxetable}
% \clearpage

% \clearpage
% \bibliographystyle{aasjournal}
% \bibliography{AGN_Feedback_Paper}


\end{document}
